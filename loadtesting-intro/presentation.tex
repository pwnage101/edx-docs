\documentclass{beamer}
\usepackage{tikz}
\usetikzlibrary{positioning} 
\usetikzlibrary{arrows.meta}

\title{edX Load Tests}
\subtitle{Introduction And Current Status}
\author{Troy Sankey}
\date{\today}

\begin{document}

\tikzstyle{title} = [text centered, text width=10em]
\tikzstyle{base} = [rectangle, draw=black, text centered, text width=10em]
\tikzstyle{mstep} = [base]
\tikzstyle{to} = [->,>={Latex[width=2mm,length=2mm]}]
\tikzstyle{to-emph} = [to,ultra thick]
\tikzstyle{to-deemph} = [to,opacity=0.3]

\begin{frame}
\titlepage
\end{frame}

\begin{frame}
\frametitle{Introduction}
\setbeamercovered{invisible}
\begin{itemize}
\item Tests live in the edx/edx-load-tests repo.\pause
\item Uses locust.io load testing framework.\pause
\item Tasks are written as python functions.\pause
% TODO: insert small example of a task
\item Tasks grouped into TaskSets.\pause
% TaskSets represent different kinds of activities a user engages in.  For
% example forums tasks are grouped inside the ForumsTasks TaskSet.
\item TaskSet nesting.
\end{itemize}
\end{frame}


\begin{frame}
\frametitle{User Experience}
\setbeamercovered{invisible}
\begin{itemize}
\item No browser, no javascript, no good measure of user experience.\pause
\item Focus on the system response, not the users.\pause
\item If you want, during a load test you could login from a browser.
\end{itemize}
\end{frame}


\begin{frame}
\frametitle{Running Load Tests Overview}
\setbeamercovered{invisible}
\begin{center}
\begin{tikzpicture}
\node[title] (rightColumn) {Target};
\node[mstep, below=1em of rightColumn] (loadtest) {Loadtest Environment};\pause
\node[title, left=of rightColumn] (leftColumn) {Source};
\node[mstep, left=of loadtest] (runLoadtest) {run-loadtest (jenkins)};
\uncover<-6>{\draw[to] (runLoadtest) -- (loadtest);}\pause
\node[mstep, below=of loadtest] (sandbox) {Sandbox};
\uncover<-6>{\draw[to] (runLoadtest) -- (sandbox);}\pause
\node[mstep, below=6.5em of runLoadtest] (workstation) {Local workstation};
\uncover<-6>{\draw[to] (workstation) -- (loadtest);}
\uncover<-6>{\draw[to] (workstation) -- (sandbox);}\pause
\node[mstep, below=of runLoadtest] (sandbox2) {Sandbox};
\uncover<-6>{\draw[to] (sandbox2) -- (loadtest);}
\uncover<-6>{\draw[to] (sandbox2) -- (sandbox);}\pause
\node[mstep, below=of sandbox] (devstack) {Devstack};
\uncover<-6>{\draw[to] (workstation) -- (devstack);}\pause

\draw[to-emph] (runLoadtest) -- (loadtest);
\draw[to-deemph] (runLoadtest) -- (sandbox);
\draw[to-deemph] (workstation) -- (loadtest);
\draw[to-deemph] (workstation) -- (sandbox);
\draw[to-deemph] (sandbox2) -- (loadtest);
\draw[to-deemph] (sandbox2) -- (sandbox);
\draw[to-deemph] (workstation) -- (devstack);

% In most cases, the run-loadtest job targetting the loadtest environment is
% what to use for large scale testing.

% Driving load against a sandbox will likely reveal most performance
% regressions, but the loadtest environment awards full-scale database,
% autoscaling groups, IDAs deployed separately, etc.

% For developing load tests you can just run locust locally and target your
% local devstack.

\end{tikzpicture}
\end{center}
\end{frame}


\begin{frame}
\frametitle{Settings}
\begin{itemize}
\item Each load test is configurable via a YAML file.\pause
% This wasn't always the case, do not revert back to reading environment
% variables!
\item Use \texttt{helpers.settings}.\pause
\item The Jenkins job (run-loadtest) will use settings specified by a job
      parameter.
\end{itemize}
\end{frame}


\begin{frame}
\frametitle{Evaluating Load Test Runs}
\begin{itemize}
\item Locust provides response time breakdown.\pause
% But this doesn't really tell us more than NR.
\item Click on the generated NewRelic links.\pause
% TODO: insert screenshot
\item NewRelic provides deeper application insight.\pause
% TODO: examples
\item Custom metrics in edx-platform: make use of the custom metrics middleware
      for peering into application behavior.
\end{itemize}
\end{frame}


\begin{frame}
\frametitle{Record Your Findings}
\begin{itemize}
\item Create a wiki page before testing.\pause
% This is a place where you should keep all data, scripts, configuration, and
% test results.  Armed with the wiki page, another person should be able to
% reproduce your test.
\item Caution: Beware of NewRelic data atrophie. NR gets hungry and eats your
      old dots.\pause
% Anything over 7 days old is of varying levels of quality.
\item Continuous deployment $\Rightarrow$ continuous deletion of system metrics!
% What instances?  They were terminated yesterday!
\end{itemize}
\end{frame}


\begin{frame}
\frametitle{Distributed load testing}
\begin{itemize}
\item Just run multiple load tests simultaneously in Jenkins.\pause
% Locust distributed mode not currently implemented in run-loadtest.  It would
% basically only award us the convenience of one click.
\item You may need to prime N workers first.\pause
\item If you want/need real distributed load testing, demand it!
% Mention that the jenkins 2.0 upgrade is pseudo-blocking any further work on
% run-loadtest job enhancements.
\end{itemize}
\end{frame}


\begin{frame}
\frametitle{How to maintain load tests}
\begin{itemize}
\item Task ratios will rot.\pause
\item Tasks themselves will rot.\pause
\item New endpoints will need to be added as new tasks.
% If you never make sure they work, you may need to fix them every time you
% need them.  I'll be working on a notification system for indicating when a
% load test becomes obviously broken.
\end{itemize}
\end{frame}


\begin{frame}
\frametitle{For fun: Real browser load testing}
\begin{itemize}
\item \texttt{pip install realbrowserlocusts}, then subclass PhantomJSLocust in
      your locustfile.
\item Could make certain tests more realistic (AJAX calls actually happen).
% E.g. javascript changed to make more AJAX calls
\item Potentially far more resource intensive on the client side
% Javascript execution on the client likely becomes a bottleneck.
\end{itemize}
\end{frame}


\end{document}
